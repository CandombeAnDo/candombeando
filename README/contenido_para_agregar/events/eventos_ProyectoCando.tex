\documentclass[a4paper]{article}
\usepackage{url} 
\usepackage{natbib}

\begin{document}

\title{Eventos Proyecto Cando} 
\author{}
\date{}
\maketitle

\subsection*{Conference/Congreso}

\begin{itemize}

\item[] 	% ictm2022
\emph{46th World Conference of the International Council for Traditional Music -- ICTM2022},
Lisboa, Portugal, 21 al 27 de julio, 2022.
Luis Jure y Martín Rocamora: presentación oral del artículo
``Micro-timing in Uruguayan Candombe Drumming''.

\item[] 	% SoMoS2022
\emph{Second Symposium of the ICTM Study Group on Sound, Movement, and the Sciences, SoMoS 2022},
Barcelona, España, 26 al 28 de octubre, 2022.
Luis Jure: presentación oral del artículo
``Documentation and analysis of Uruguayan candombe drumming''.

\item[] 	% aawm2022
\emph{Seventh International Conference on Analytical Approaches to World Music, 2022},
Sheffield, Reino Unido, 14 al 17 de junio, 2022.
Luis Jure: presentación oral (a distancia) del artículo
``The piano drum style of Gustavo Oviedo, candombe master drummer''.

\item[] 	% aawm2021
\emph{Sixth International Conference on Analytical Approaches to World Music (AAWM 6)},
París, Francia, 9 al 12 de junio, 2021.
Luis Jure: presentación oral (a distancia) del póster
``Tempo, Micro-tempo and Dynamics in Uruguayan Candombe Drumming''.

\item[]	% aawm2019
\emph{First Analytical Approaches to World Music Special Topics Symposium},
Birmingham, Reino Unido, 2 al 4 de julio, 2019.
Luis Jure: presentación oral del artículo
``\texttt{carat}: A toolbox for computer--aided rhythm analysis''.

\item[]	% fma2018
\emph{8\textsuperscript{th} International Workshop on
Folk Music Analysis -- FMA 2018},
Thessaloniki, Grecia, 26 al 29 de junio, 2018.
Luis Jure: presentación oral del artículo ``\emph{Subir la llamada}:
Negotiating tempo and dynamics in Uruguayan Candombe drumming''.

\item[]	% aawm2018
\emph{Fifth International Conference on Analytical Approaches
to World Music -- AAWM 2018},
Thessaloniki, Grecia, 26 al 29 de junio, 2018.
Luis Jure: presentación oral del artículo ``Improvisation techniques of the
\emph{repique} drum in Uruguayan Candombe drumming''.

\item[] 	% RRPW 2017
\emph{16\textsuperscript{th} Rhythm Production and Perception Workshop -- RPPW 2017},
Birmingham, Inglaterra, 3 al 5 de julio, 2017.
Luis Jure: presentación por póster del artículo ``Clave patterns in Uruguayan Candombe drumming''.

\item[]	% ICTM 2017
\emph{44\textsuperscript{th} International Council for Traditional Music World Conference -- ICTM 2017},
Limerick, Irlanda, 13 al 19 de julio, 2017.
Luis Jure: presentación oral del artículo ``Timeline patterns in Uruguayan Candombe drumming''.

\item[] 	% aawm2016
\emph{Fourth International Conference on Analytical Approaches
to World Music -- AAWM 2016},
Nueva York, Estados Unidos, 8 al 11 de junio, 2016.
Luis Jure: presentación oral del artículo ``Microtiming in the Rhythmic
Structure of Candombe Drumming Patterns''.

\item[] 	% ismir2015
\emph{16\textsuperscript{th} International Society for Music Information Retrieval
Conference -- ISMIR 2015}, Málaga, España, 26 al 30 de octubre, 2015.
Luis Jure y Martín Rocamora: presentación oral del artículo
``Beat and Downbeat Tracking Based on Rhythmic Patterns
Applied to the Uruguayan Candombe Drumming''.

\item[]	% cictem2015
\emph{Congreso Internacional de Ciencia y Tecnología Musical -- CICTeM 2015},
Buenos Aires, Argentina, 17 al 19 de septiembre, 2015.
Luis Jure y Martín Rocamora: presentación oral del artículo
``An audio--visual database of Candombe performances for computational
musicological studies''.

\item[]	% cim2014
\emph{9\textsuperscript{th} Conference on Interdisciplinary Musicology -- CIM14}.
Berlín, Alemania, 4 al 6 de diciembre, 2014.
Luis Jure y Martín Rocamora: presentación oral del artículo
``Tools for detection and classification of piano drum
patterns from candombe recordings''.

\end{itemize}



\subsection*{Keynote}

\begin{itemize}

\item[]	% XXXX
``Some Title"
Presentación en ``Interpersonal Entrainment in Music Performance Workshop''.
Some place, some date.\\
Martín Rocamora.

\end{itemize}

\subsection*{Colloquium/Coloquio}

\begin{itemize}

\item[]	% coloquio 2021
``Proyecto \emph{`Documentación y Análisis del Candombe Uruguayo'}:
un enfoque desde la musicología computacional''.
Presentación en el ``Encuentro de Investigación en Música y Sonido''.
Escuela Universitara de Música, Universidad de la República.
Montevideo, Uruguay, 15 de octubre, 2021.\\
Luis Jure.

\item[ ]	% coloquio2011
``Principios generativos del toque del repique del Candombe''.
Presentación en el coloquio \emph{La música entre África y América}, Centro Nacional de Documentación
Musical Lauro Ayestarán del Ministerio de Educación y Cultura.
Montevideo, Uruguay, 3 de octubre, 2011.\\
Luis Jure.

\item[ ]	% columbia2010
``Drums and the City. Musical traits and performance
practices of Candombe in Montevideo''.
Presentación en el coloquio \emph{Sounding the Space}.
Columbia University. 
Nueva York, Estados Unidos, 8 de octubre, 2010.\\
Luis Jure.

\end{itemize}

\subsection*{Workshop}

\begin{itemize}

\item[]	% IEMP2018
``Some insights/perspectives on interpersonal
entraniment in candombe drumming''.
Presentación en el ``Interpersonal Entrainment in Music Performance Workshop''.
Génova, Italia, 8 de marzo, 2018.\\
Luis Jure, Martín Rocamora.

\item[]	% IEMP2016
``An audio--visual database of Candombe performances
for computational musicological studies''.
Presentación en el ``Interpersonal Entrainment in Music Performance Workshop''.
Durham, Reino Unido, 15 de mayo, 2016.\\
Luis Jure, Martín Rocamora.

\end{itemize}

\subsection*{Talk/Conferencia}

\begin{itemize}

\item[ ]	% MDW2022
Institut für Volksmusikforschung und Ethnomusikologie, Universität für Musik und darstellende Kunst Wien.
``Musical Traits and Performance Practice of Uruguayan Candombe Drumming:
A Computational Musicological Approach''.
Viena, Austria, 19 de octubre, 2022.\\
Luis Jure.

\item[ ]	% MaxPlanck2022
Max Planck Institute for Empirical Aesthetics.
``A computational approach to Uruguayan Candombe drumming''.
Fránfort del Meno, Alemania, 10 de agosto, 2022.\\
Luis Jure, Martín Rocamora.

\item[ ]	% stanford2018
Stanford University, Center for Computer Research in Music and Acoustics -- CCRMA.
``Musical Traits and Performance Practice of Uruguayan Candombe Drumming:
A Computational Musicological Approach''.
Palo Alto, California, Estados Unidos, 14 de noviembre, 2018.\\
Luis Jure.

\item[ ]	% ucdavis2018
University of California, Davis, Department of Music.
``Musical Traits and Performance Practice of Uruguayan Candombe Drumming:
A Computational Musicological Approach''.
Davis, California, Estados Unidos, 8 de noviembre, 2018.\\
Luis Jure.

\item[ ]	% columbia2017
Columbia University, Computer Music Center.
``Uruguayan Candombe drumming -- Musical traits and performance practice:
A case study for computational musicology''.
Nueva York, Estados Unidos, 13 de diciembre, 2017.\\
Luis Jure.

\item[ ]	% tu2017
Technische Universität Berlin, Fachgebiet Audiokommunikation.
``Uruguayan Candombe drumming -- Musical traits and performance practice:
A case study for computational musicology''.
Berlín, Alemania, 11 de abril, 2017.\\
Luis Jure.

\item[ ]	% ofai2014
Österreichisches Forschungsinstitut für Artificial Intelligence -- OFAI.
``Tools for detection and classification of piano drum patterns from Candombe recordings''.
Viena, Austria, 12 de diciembre, 2014.\\
Luis Jure.

\item[ ]	% ofai2013
Österreichisches Forschungsinstitut für Artificial Intelligence -- OFAI.
``Rhythmic pattern analysis: the Uruguayan Candombe drumming as a case study''
Viena, Austria, 19 de noviembre, 2013.\\
Luis Jure.

\item[ ]	% cictem2013
\emph{Congreso Internacional de Ciencia y Tecnología Musical -- CICTeM 2013}.
``Detección automática de patrones rítmicos: el Candombe uruguayo como caso de estudio''.
Buenos Aires, Argentina, 26 al 28 de septiembre, 2013.\\
Luis Jure, Martín Rocamora.

\end{itemize}

\subsection*{Outreach/Extensión}

\begin{itemize}

\item[]	% IEMP2018
``Registros de Ansina''.
Presentación en el taller ``Ansina Tradición -- Transmisión, no imposición''.
Casa de la Cultura Afrouruguaya 
Montevideo, Uruguay 4 de mayo, 2019.\\
Luis Jure.

\end{itemize}

\end{document}

